\documentclass{article}
\usepackage[utf8]{inputenc}

\title{\textbf{FOAR705 Scoping Project}}
\author{\textbf{Emily Hunt 44888619}}
\date{August 2019}

\begin{document}

\maketitle

\section*{Jobs To Do While Completing Thesis}
Scope: The entire thesis writing process including data collection, data analysis, composition of thesis and publication.
\begin{itemize}
    \item Plan the overall essay structure.
    \item Search the university's online library and within a number of journals for relevant sources.
    \item Search through a number of newspaper and magazine databases for relevant film reviews, and then read them for their relevance.
    \item Analyse relevant films, considering, for instance, their plot, themes, and colour palette.
    \item My process involves reading the sources which have been collected, finding relevant quotes and copying these into a word document. I then create a key using highlights to indicate their sources, re-order and refine the quotes if they are useful, use them in my writing, and then retrace their sources from the original document.
    \item Go back through various documents to find any quotes or information which may have been edited out of my writing, but that I want to re-insert.
    \item Read through the document and fill in any incomplete references.
    \item Format my thesis appropriately for submission.
    \item Reformat it if necessary to submit it to other places, such as journals.
    \item Go through the document to ensure every in-text reference is included within the bibliography, and every bibliographic entry is referenced in-text.
    \item Manually ensure all in-text references and bibliography entries are in the appropriate style, and referenced correctly.
\end{itemize}

\pagebreak

\section*{Pains}
\subsection*{Pains I Am Likely to Encounter}
\begin{itemize}
    \item My current methods of searching the university's online library for relevant sources and searching relevant databases, can be a really long and sometimes unreliable process.
    \item Spending a large amount of time adjusting my document to be in the correct format for a thesis.
    \item Tediously changing the entire document each time a new format is required.
    \item Undertaking the frustrating process of searching each individual bibliography entry to ensure there is a corresponding reference in the text, as well as the reverse, with the potential of missing references.
    \item Changing each individual in-text reference and bibliography entry every time a new referencing style is required, and making sure they are correct within each individual style.
    \item Determining the source of quotes/references I have included in my writing, and finding quotes which may have been cut.  
\end{itemize}

\subsection*{Pain Relievers}
It would be really useful to have a tool that:
\begin{itemize}
    \item Allows me to more efficiently search within journal, newspaper and magazine databases for relevant and useful material. 
    \item Helps me quickly and simply put my thesis in an appropriate  format, and also switch to other formats that may be potentially required.
    \item Allow me to check each in-text references has a corresponding bibliography entry, as well as the reverse, or potentially generates in-text references from my bibliography.
    \item Can correctly format my references, and allow me to easily switch them to a different style. It would also be useful if this tool was easy to update, so I could add references consistently from the beginning of my research process to avoid missing any. 
    \item Keeps a record of the information/quotes I have used and its source.
\end{itemize}

\pagebreak

\section*{Gains}
\subsection*{Gains I Would Like to Make}
It would be great if I could have a way of:
\begin{itemize}
    \item More clearly planning my essay structure, that I could add notes to without it becoming difficult to read. This is in contrast to my current method of writing bullet points in a word document, which after I continue to add to, become difficult to read and overwhelming.
    \item Connecting notes I take on a film to the film itself that is more efficient than taking note of a time stamp and scrubbing through the film.
    \item Quantitatively measuring the colours of a visual image or a film, which I could then infer the artistic meaning of.
    \item Organising my notes that I currently keep across numerous word documents (e.g. having quotes from journal articles, notes I have taken on films, and notes from books all in separate files) in a more clear, manageable and efficient manner.
\end{itemize}

\subsection*{Gain Creators}
It would be really useful to have a tool that:
\begin{itemize}
    \item Allows me to visually map out my essay structure, which I could add notes, links, references etc. to, without cluttering the basic structure overview. 
    \item Is able to link notes to a specific frame/section of a film. It would be great if I could designate different kinds of notes (e.g. notes related to themes, framing, movement etc.) which I would then be able to search for.
    \item Analyses the colour composition of an image, short clip or an entire film. It would be great if this tool allowed me to compare the colour data of two different shots/pieces of film.
    \item Collated the information I currently have in numerous places in a more clear manner than simply placing it all in the same word document, which makes it simple to sort and search the data.
\end{itemize}




\end{document}
